\documentclass[12pt]{article} %,fleqn
\usepackage{mtaithesis}

\addbibresource{mtaibiblio.bib}

\begin{document}

{
\renewcommand{\baselinestretch}{1}
\thispagestyle{empty}
\begin{center}
    \sc
        Министерство науки и высшего образования Российской Федерации\\
        Московский физико-технический институт\\
        {\rm(национальный исследовательский университет)}\\
        Физтех-школа прикладной математики и информатики\\
        Магистерская программа\\
        <<Методы и технологии искусственного интеллекта>>\\[35mm]
    \rm\large
        Сотников Антон Дмитриевич\\[10mm]
    \bf\Large
        Байесовский выбор архитектуры нейросетевой модели\\[10mm]
    \rm\normalsize
        03.04.01 --- Прикладные математика и физика\\[10mm]
    \sc
        Магистерская диссертация\\[30mm]
\end{center}
\hfill\parbox{80mm}{
    \begin{flushleft}
    \bf
        Научный руководитель:\\
    \rm
        к ф.-м. н\\
        Бахтеев Олег Юрьевич\\[5cm]
    \end{flushleft}
}
\begin{center}
    Москва\\
    2022 г.
\end{center}
}

\newpage
\tableofcontents

\newpage
\begin{abstract}
   В работе исследуется задача выбора структуры модели нейронной сети. Вводятся априорные распределения на параметры, гиперпараметры и структуру модели. Предполагается зависимость параметров от структуры модели.  Предлагается метод, вычисляющий апостериорного совместного распределения структуры и параметров модели с помощью байесовского вывода. В силу практической невычислимости такого вывода распределение предлагается оценивать с помощью оптимизации вариационной нижней оценки.


\end{abstract}

\newpage
%%%%%%%%%%%%%%%%%%%%%%%%%%%%%%%%%%%%%%%%%%%%%%%%%%%%%%%%%%%%%%%%%%%%%%%%%%
\section{Введение}
%%%%%%%%%%%%%%%%%%%%%%%%%%%%%%%%%%%%%%%%%%%%%%%%%%%%%%%%%%%%%%%%%%%%%%%%%%



%%%%%%%%%%%%%%%%%%%%%%%%%%%%%%%%%%%%%%%%%%%%%%%%%%%%%%%%%%%%%%%%%%%%%%%%%%
\section{Обзорно-постановочный раздел работы}
%%%%%%%%%%%%%%%%%%%%%%%%%%%%%%%%%%%%%%%%%%%%%%%%%%%%%%%%%%%%%%%%%%%%%%%%%%

\subsection{Основные понятия и определения}
Вводятся общепринятые понятия и~обозначения со~ссылками на~литературу.

\subsection{Обзор современного состояния проблемы}


\newpage

\nocite{*}
\printbibliography[title={Список литературы}]

\end{document}
